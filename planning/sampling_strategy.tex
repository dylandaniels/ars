\documentclass{article}
\usepackage{amsmath}

\begin{document}

Explanation of algorithm for sampling from $s_k(x) = u_k(x) / \int_D u_k(x') dx' = c \ u_k(x)$. First we need to find the normalizing constant $c := 1 / \int_D u_k(x') dx':$

$$
\int_D u_k(x') dx' = \sum_{j = 1}^k \int_{z_{j-1}}^{z_j} \exp(h(x_j) + (x' - x_j) h'(x_j)) dx'
$$
$$
 = \sum_{j = 1}^k \frac{1}{h'(x_j)} \exp(h(x_j) + (x' - x_j) h'(x_j)) \bigg\rvert^{z_j}_{x' = z_{j - 1}}
$$
$$
 = \sum_{j = 1}^k \frac{exp(u_k(z_j)) - exp(u_k(z_{j-1}))}{h'(x_j)}
$$

Next, in order to use the inverse CDF method for sampling, we must find the CDF for $s_k(x)$, $S_k(x) = c \int_{z_0}^x u_k(x') dx'$:

$$
S_k(x) = c \left(\sum_{j = 1}^{t - 1} \int_{z_{j-1}}^{z_j} \exp(h(x_j) + (x' - x_j) h'(x_j)) dx' + \int_{z_{t-1}}^x \exp(h(x_t) + (x' - x_t) h'(x_t)) dx') \right)
$$
$$
 = c \left( \sum_{j = 1}^{t-1} \frac{u_k(z_j) - u_k(z_{j-1})}{h'(x_j)} + \frac{u_k(x) - u_k(z_{t-1})}{h'(x_t)} \right)
$$

\noindent where $t$ is the index of which interval of $z$'s that $x$ lies in. Formally, it is $t(x) = i : x \in (z_{i-1}, z_i)$. For convenience, we call the sum term above \texttt{partialSums[t-1]}. Note that our normalizing constant $c =$ \texttt{1/partialSums[k]}. Thus,

$$
S_k(x) = c \left(\text{\texttt{partialSums[t-1]}} + \frac{u_k(x) - u_k(z_{t-1})}{h'(x_t)} \right)
$$

Now we can determine the inverse transform $S_k^{-1}(U)$, where $U \sim$ Uniform[0,1]. Because $t$ is actually a function of $x$, it too must be inverted. Intuitively, we want to pick the biggest $t$ such that $S_k(z_{t-1}) < U$. Formally, $t(U) = i : U \in (c\ \text{\texttt{partialSums[i - 1]}}, c\ \text{\texttt{partialSums[i]}})$. Solving for the inverse, we have:

$$
S_k^{-1}(U) = \frac{\log\left(h'(x_t) (\frac{U}{c} - \text{\texttt{partialSums[t-1]}}) + u_k(z_{t-1})\right) - h(x_t)}{h'(x_t)} + x_t
$$


\end{document}