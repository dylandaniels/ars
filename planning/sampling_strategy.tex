\documentclass{article}
\usepackage{amsmath}

\begin{document}

Explanation of algorithm for sampling from $s_k(x) = u_k(x) / \int_D u_k(x') dx' = c \ u_k(x)$. First we need to find the normalizing constant $c := 1 / \int_D u_k(x') dx':$

$$
\int_D u_k(x') dx' = \sum_{j = 0}^{k+1} I_j,
$$

where $$I_j := \int_{z_{j-1}}^{z_j} \exp(h(x_j) + (x' - x_j) h'(x_j)) dx'.$$ Notice that the explicit form of $I_j$ depends on whether $h'(x_j)=0$ or not. Therefore we have,
$$ I_j = \begin{cases} 
	(z_j-z_{j-1}) \exp (h(x_j)) \quad \text{if} \quad h'(x_j) = 0, \vspace{2mm} \\ 
	\frac{\exp u_k(z_j)  - \exp u_k(z_{j-1}) }{h'(x_j)} \quad \text{otherwise}.
	\end{cases} $$ 

Next, in order to use the inverse CDF method for sampling, we must find the CDF for $s_k(x)$, $S_k(x) = c \int_{z_0}^x u_k(x') dx'$:

$$
S_k(x) = c \left(\sum_{j = 0}^{t - 1} I_j + \int_{z_{t-1}}^x \exp(h(x_t) + (x' - x_t) h'(x_t)) dx') \right),
$$

\noindent where $t$ is the index of which interval of $z$'s that $x$ lies in. Formally, it is $t(x) = \{1\leq i \leq k+1 : x \in (z_{i-1}, z_i)\}$. For convenience, let $$\texttt{partialSums[t-1]} := \sum_{j = 1}^{t - 1} I_j$$ and notice that our normalizing constant can be expressed as $c = \texttt{1/partialSums[k]}$. Moreover, let us define $$J_{t-1}(x) := \int_{z_{t-1}}^x \exp(h(x_t) + (x' - x_t) h'(x_t)) dx').$$ Then we have,
$$
S_k(x) = c \left(\text{\texttt{partialSums[t-1]}} + J_{t-1}(x) \right),
$$
where 
$$
J_{t-1}(x) = \begin{cases} 
	\exp\big(h(x_t)\big)(x-z_{t-1}) \quad \text{if} \quad h'(x_t) = 0, \vspace{2mm}\\
 	\frac{\exp u_k(x) - \exp u_k(z_{t-1})}{h'(x_t)} \quad \text{otherwise}.
 \end{cases}
$$


Now, we can determine the inverse transform $S_k^{-1}(U)$, where $U \sim$ Uniform[0,1]. Because $t$ is actually a function of $x$, it too must be inverted. Intuitively, we want to pick the biggest $t$ such that $S_k(z_{t-1}) < U$. Formally, $t(U) = \{ 1\leq i \leq k+1 : U \in (c \times \text{\texttt{partialSums[i - 1]}}, c\times \text{\texttt{partialSums[i]}})\}$. Solving for the inverse, we have:

$$
S_k^{-1}(U) = \begin{cases}
	\frac{\frac{U}{c} - \texttt{partialSums[t-1]}}{\exp \big(h(x_t) \big)} + z_{t-1} \quad \text{if} \quad h'(x_t) = 0, \vspace{2mm} \\
 	\frac{\log\left(h'(x_t) (\frac{U}{c} - \text{\texttt{partialSums[t-1]}}) + \exp u_k(z_{t-1})\right) - h(x_t)}{h'(x_t)} + x_t \quad \text{otherwise}.
 \end{cases}
$$


\end{document}